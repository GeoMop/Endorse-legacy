\documentclass[a4paper]{article}

\usepackage{amsthm}
\usepackage{amsmath,amssymb}
\usepackage{graphicx}
\usepackage[cm]{fullpage}
\usepackage{color}

\setlength{\oddsidemargin}{-1mm}
\setlength{\topmargin}{0cm}
\setlength{\textwidth}{16cm}
\setlength{\textheight}{22.5cm}
\parskip 1.5ex
\parsep 1.5ex
\parindent 0mm


\newtheorem{theorem}{V\v{e}ta}
%\newtheorem{corollary}{Důsledek}
\newtheorem{lemma}{Lemma}
%\theoremstyle{definition}
%\newtheorem{definition}{Definice}
%\newtheorem{remark}{Poznámka}
%\newtheorem{example}{Příklad}

%\numberwithin{equation}{section}
%\numberwithin{definition}{section}
%\numberwithin{lemma}{section}
%\numberwithin{remark}{section}
%\numberwithin{corollary}{section}
%\numberwithin{theorem}{section}
%\numberwithin{example}{section}
%\numberwithin{figure}{section}

\newcommand{\alertg}[1]{{\color{green}#1}}
\newcommand{\alert}[1]{{\color{red}#1}}
\newcommand{\alertb}[1]{{\color{blue}#1}}


\newcommand{\mbf}[1]{\mbox{\boldmath$#1$}}
\title{Definice vstupních dat a indikátorů bezpečnostních funkcí EDZ}


\author{RB and coworkers}

\begin{document}
	
\maketitle
	
\tableofcontents
	
% \section{Úvod}
% 
% Oblast porušení horninového masivu v okolí podzemních děl, známá jako Excavation
% damaged zone (EDZ)\footnote{reference}, je považována za důležitou součást návrhů
% hlubinného uložistě vyhořelého jaderného paliva. Důvodem je možnost vytvoření cesty, která má větší hydraulickou vodivost ve srovnání s okolním horninovým prostředím a může tak oslabit izolační schopnost horninové bariéry proti úniku škodlivých látek do biosféry. Důležitost EDZ je všeobecně
% akceptována \cite{1}, a existuje řada experimentálních i teoretických prací, které se snaží o charakterizaci EDZ z hlediska mechanického poškození a návazných hydraulických vlastností\footnote{reference}.
% 
% Pro hlubší poznání EDZ se v projektu TAČR ENDORSE zaměříme na
% matematické modelování s využitím experimentálních měření pro získání
% vlastností EDZ a vstupních parametrů pro následné matematické modely. Matematické modely pak umožňují simulaci probíhajících procesů, posouzení scénářů šíření kontaminantů i odhad citlivosti simulací na různé
% parametry a následně tak i optimalizaci měření nebo monitoringu.
% 
% Tato zpráva je částečně rešeršní, ale především popisuje kroky, které lze
% použít pro specifikaci dalšího výzkumu v rámci projektu TAČR ENDORSE
% a vytvoření metodiky pro predikci indikátorů bezpečnosti EDZ na základě
% geofyzikálních měření a matematického modelování.


% {\bf Ad c)}\\
% Pro výpočet poškození hornin je vhodné uvažovat hierarchii modelů \cite{13}, se zvětšující se složitostí a také různou potřebou vstupních parametrů:
% \begin{itemize}
% 	\item[(i)] elastický výpočet napětí a deformací a následně indikace, kde dochází k porušení materiálu podle vhodného pevnostního kritéria, 
% 	\item[(ii)] výpočet napětí a deformací za předpokladu, že nemůže dojít ke stavům za mezí platnosti vhodného pevnostního kritéria (plastické modely). Lze uvažovat procesy s postupným zatěžováním a monotónní odezvou (deformační teorie plasticity) i modely se zatěžováním a odlehčováním a trvalými (plastickými) deformacemi,
% 	\item[(iii)] výpočet napětí a deformací za předpokladu, že nemůže dojít ke stavům za mezí platnosti vhodného pevnostního kritéria a zároveň dochází k oslabení (porušování,  damage) materiálu (modely  spojité mechaniky porušení, continuum damage mechanics). V případě samotné damage mechanics se neuvažují nevratné deformace.
% \end{itemize}
% U modelů (i) a (ii) máme většinou řadu informací o existenci a jednoznačnosti
% řešení, o stabilitě, tedy spojité závislosti řešení na vstupních datech, o chybě vznikající  při  diskretizaci  úloh  a  pod.  Složitější je porozumění  modelům
% continuum damage mechanics (CDM), pro které existuje závislost diskretizovaných  modelů na  síti, nejednoznačnost řešení  apod. Proto jsou navrženy CDM modely s regularizací nebo s nelokální definicí deformací a napětí.
% \begin{itemize}
% 	\item[(iv)] V praxi existuje řada kombinací výše uvedených modelů. Jednak lze uvažovat kombinaci plasticity a damage mechanics, která uvažuje jak oslabení, tak i trvalé deformace. Existuje však také mnoho ad-hoc přístupů, které uvažují postupné oslabování materiálových konstant, viz např. \cite{18,5}.
% \end{itemize}
% 
% Každý typ modelu vyžaduje specifické parametry a našim cílem bude jak popsat model, tak následně uvažovat citlivost na změny jednotlivých parametrů i možnost, jak parametry určit. Standardním způsobem určení parametrů jsou laboratorní zkoušky \alert{se vzorky hornin. Ty jsou ale rozměrově odlišné a více intaktní narozdíl od horninového masívu.} Vítaná je tedy možnost získání parametrů pomocí inverzních úloh a in situ měření.

% \subsection{Modely pro výpočet mechanického porušení hornin} 
% 
% Základním modelem pro výpočet napětí a deformací je lineární elasticita s tenzorem malých deformací a Cauchyovým tenzorem napětí, která je popsána trojicí posunutí $u$, deformace $\varepsilon$ a napětí $\sigma$ a vzájemnými vztahy, které platí ve výpočetní oblasti $\Omega$:
% \begin{eqnarray*}
% 	- div(\sigma) &=& f \quad \mbox{v} \ \Omega, \\
% 	\sigma &=& C : \varepsilon \quad  \mbox{v} \ \Omega, \\
% 	\varepsilon(u) &=& (\nabla u + (\nabla u)^\top)/2 \quad \mbox{v} \ \Omega.
% \end{eqnarray*}
% Výše, $f$ vyjadřuje objemové síly (hustotu) a $C$ je elastický tenzor pružnosti. Dále na hranici $\partial \Omega$ musíme zadat vhodné okrajové podmínky. V uvedené formulaci jsou kompresní deformace a normálová napětí v tlaku záporná \alert{(běžná mechanická znaménková konvence)}. Pro popis pevnostních veličin se ale v geotechnice užívá opačná konvence, kdy normálová napětí v tlaku jsou kladná. Potom budeme uvažovat hlavní napětí $\sigma_3 \leq \sigma_2 \leq \sigma_1$, kde $\sigma_1$ je \alert{největší hlavní napětí, které je obvykle kladné v tlaku.}
% 
% Poznamenejme  ještě, že počáteční napětí $\sigma_0$, které může být určeno měřením \alert{nebo výpočtem pomocí inverzní analýzy (viz zaslaný článek od J. Malíka),  může být zavedeno do konstitučního vztahu následovně:} $\sigma = C  : \varepsilon + \sigma_0$ v $\Omega$. Obdobně může být do konstitučního vztahu zaveden pórový tlak podzemní vody.
% 
% {\bf Ad (i):}\\
% Pevnostní kritéria vychází z vypočteného napětí, případně deformace. Přípustné stavy je možné popsat nerovností
% $$
% 	F_s = F_s(\sigma, \varepsilon) \leq 0.
% $$
% Kritérii často používanými v mechanice hornin jsou Mohrovo-Coulombovo a  Hoekovo-Brownovo  kritérium, viz např. \cite{7, 8,10} a Brady, Brown 2006 \cite{4}. Tyto kritéria nyní popíšeme za předpokladu isotropní horniny.
% \begin{itemize}
% 	\item Mohrovo-Coulombovo kritérium
% 		$$
% 			F (\sigma, x) = |\tau| - (c + \sigma_n \mbox{tg} (\phi)) ,
% 		$$
% 		kde $\tau$ a $\sigma_n$ je smykové a normálové napětí působící na libovolnou plochu procházející bodem x, $c$ je koheze a $\phi$ je úhel vnitřního tření.
% 	\item Mohrovo-Coulombovo kritérium lze také vyjádřit pomocí hlavních napětí. Kritické napětí se týká plochy, která je rovnoběžná se směrem prostředního hlavního napětí. Pokud její polohu vyjádříme úhlem $\beta$ mezi normálou a směrem $\sigma_1$, potom na ni budou působit napětí
% 		$$
% 			\sigma_n = \frac{1}{2}(\sigma_1 + \sigma_3 ) + \frac{1}{2}(\sigma_1 - \sigma_3 )\cos 2 \beta, \quad \tau = \frac{1}{2}(\sigma_1 - \sigma_3 ) \sin 2 \beta.
% 		$$
% 		Kritický poměr pak nastane pro $\beta =  \frac{\pi}{4} + \frac{\varphi}{2}$. Pro tento úhel dostaneme
% 		$$
% 			F(\sigma,x) = (\sigma_1 - \sigma_3 ) - \frac{2c\,\cos \phi}{1-\sin \phi} - \frac{2\sin \phi}{1-\sin \phi} \sigma_3 .
% 		$$
% 		Kritérium lze také vyjádřit vztahem
% 		$$
% 			F(\sigma, x) = (\sigma_1 - \sigma_3) - (q_{cmass} + A\sigma_3) \leq 0
% 		$$
% 		kde $q_{cmass}$ je jednoosá pevnost v tlaku (UCS) pro masiv, $q_{cmass} = 2c \cos \phi / (1-\sin \phi)$ a $A = 2 \sin \phi / (1-\sin \phi)$. UCS lze interpretovat jako Uniaxial Compressive Strength (značí se obvykle $\sigma_c$), nebo Unconfined Compressive Strength ($q_c$). Lze také využít pevnost v tahu (tensile strength $\sigma_t$),
% 		$$
% 			q_{cmass} = \sigma_c = \frac{2c \cos \phi}{1-\sin \phi}, \quad \sigma_t = \frac{2c \cos \phi}{1+ \sin \phi}.
% 		$$
% 		Další možností je vyjádření kritéria ne pomocí hlavních napětí, ale pomocí invariantů napětí. Viz např. (Desai, Siriwardane \cite{7}).
% 	\item Mohrovo-Coulombovo kritérium bývá také doplněno podmínkou žádného či částečného nepřipuštění tahových napětí, tension cut-off.
% 	\item Hoekovo-Brownovo kritérium je možno považovat za zobecnění Mohrova-Coulombova kritéria. Toto zobecnění bylo určeno empiricky a zohledňuje kvalitu hornin s 						uvažováním jednoosé pevnosti v tlaku (UCS) pro neporušenou horninu $q_c$ na jedné straně, ale parametry charakterizující kvalitu (porušenost) hornin, na straně druhé. 					Kritérium má tvar
% 		$$
% 			F(\sigma,x)=(\sigma_1 - \sigma_3 ) - q_c \left[m_b \frac{\sigma_3}{q_c} + s \right]^a ,
% 		$$
% 		kde $q_{c}$ jednoosá pevnost v tlaku neporušených kusů horniny, $m_b$ je nelineární parametr závisející na typu horniny, $a$ je parametr rozpukání horniny.
% \end{itemize}
% 
% {\bf Ad (ii):}\\
% \alert{Perfektně} plastický model je ve tvaru
% \begin{table}[ht]
% 	\centering
% 	\begin{tabular}{ll}
% 		\hline
% 		rozklad tenzoru deformace & $\varepsilon = \varepsilon^e + \varepsilon^p$\\
% 		Cauchyho napětí & $\sigma = C : \varepsilon^e$\\
% 		\alert{plastické kritérium} & \alert{$f_p(\sigma)\leq0$}\\
% 		plastický potenciál & $g_p,\;\;$ \alert{pro asociovanou plasticitu platí} $\partial g_p/\partial \sigma = \partial f_p/\partial \sigma$  \\
% 		plastické tečení & \alert{$\dot{\varepsilon}^p = \dot{\gamma}(\partial g_p/\partial \sigma), \dot\gamma \geq 0$,}\\
% 		podmínka duality & $\dot{\gamma} f_p =0$\\
% 		\hline
% 	\end{tabular}
% \end{table}
% 
% \begin{itemize}
% 	\item V případě Mohrovy-Coulombovy plasticity máme
% 	\begin{eqnarray*}
% 		f_p(\sigma) &=& \frac{1}{2} (\sigma_1 - \sigma_3) + \frac{1}{2} (\sigma_1 + \sigma_3)\sin \phi - c \cos \phi, \\
% 		g_p(\sigma) &=& \frac{1}{2} (\sigma_1 - \sigma_3) + \frac{1}{2} (\sigma_1 + \sigma_3)\sin \psi,
% 	\end{eqnarray*}
% 	\alert{kde $\psi$ úhel dilatace (dilation angle), viz \cite{9}. Poznamenejme, že pro $\phi = \psi$ platí $\partial g_p/\partial \sigma = \partial f_p/\partial \sigma$, a tedy se jedná o asociovaný model. Dále, plastické kritérium $f_p(\sigma)\leq0$ je ekvivalentní s výše uvedeným kritériem, což odvodíme přenásobením funkce
% 	$F(\sigma, x) = (\sigma_1 - \sigma_3) - \frac{2c\cos \phi}{1-\sin \phi} - \frac{2\sin \phi}{1-\sin \phi} \sigma_3$ výrazem $\frac{1}{2}(1 - \sin \phi)$.}
% 	\item Místo Mohrovy-Coulombovy plasticity lze také použít Druckerův-Pragerův model, jehož popis i vztah k parametrům Mohrovy-Coulombovy plasticity lze najít v \cite{9}. 
% 	\item Plasticita, vycházející z Hoekova-Brownova kritéria je málo obvyklá, ale možná. Viz např. \cite{5}.
% \end{itemize}
% 
% 
% {\bf Ad (iii), (iv)}
% Samotné modely plasticity neuvažují oslabení materiálu při křehkém porušení. K tomu účelu můžeme zavést parametr oslabení (damage parametr) $\omega\in[0,1]$, kde $\omega = 1$ odpovídá neporušenému materiálu a $\omega = 0$ znamená zcela porušený materiál. Nejjednodušší kombinace elasticity a porušení by potom využívala vztahy
% \begin{table}[h!]
% 	\centering
% 	\begin{tabular}{ll}
% 		\hline
% 		zobecněný Hookeův zákon & $\sigma = (1-\omega)C : \varepsilon$\\
% 		vývoj porušení & $\omega = g(\kappa), \dot{\kappa} \geq 0, \kappa(0) = \overline{\varepsilon}_0$\\
% 		funkce porušení & $f_d(\varepsilon,\kappa)=\overline{\varepsilon}(\varepsilon)-\kappa$\\
% 		podmínka přípustnosti & $f_d(\varepsilon,\kappa)\leq 0$ \\
% 		podmínku duality & $\dot{\kappa} f_d =0$\\
% 		\hline
% 	\end{tabular}
% \end{table}
% 
% Výše $\overline{\varepsilon}(\varepsilon)$ představuje ekvivalentní tahovou deformaci (Mazars)
% $$
% 	\overline{\varepsilon}(\varepsilon) = (\langle\varepsilon_1\rangle^2+\langle\varepsilon_2\rangle^2+\langle\varepsilon_3\rangle^2)^{1/2},
% $$
% kde $\langle\varepsilon_i\rangle$ značí absolutní hodnotu tahových hlavních deformací, deformace v
% tlaku se neuvažují. Všimněme si souvislosti s pevnostními kritérii využívajícími deformace, přehled lze nalézt v (Kwasniewski, Takahashi 2010). Parametr porušení se vyvíjí v závislosti na zavedené ekvivalentní tahové deformaci. Funkci $g$ můžeme volit exponenciálně s adaptací na diskretizační parametr.
% 
% Modelování porušení (Continuum Damage Mechanics) je poměrně složité, viz např. (Lemaitre 1992), (Souza Neto, Peric, Owen 2008). Proto pouze zmíníme některé složitější podrobnosti:
% \begin{itemize}
% 	\item porušení nemusí být isotropní a může být také odlišné pro namáhání v tlaku a tahu,
% 	\item v reálných případech se při porušení objevují trvalé deformace, takže
% 	použijeme opět dělení deformací $\varepsilon=\varepsilon^e + \varepsilon^p$, $\overline{\sigma} = C : \varepsilon^e$, $\sigma = (1-\omega)\overline{\sigma} = C : \varepsilon^e$. Navíc propojíme růst poškození s přírůstkem plastické deformace, $\dot{\kappa} = \|\dot{\varepsilon}^p\|^2 = \dot{\varepsilon}^p : \dot{\varepsilon}^p$.
% 	\item modely poškození je nutné stabilizovat pro zamezení závislosti na diskretizaci.
% \end{itemize}

\subsection{Mechanické chování a modelování poruch a trhlin}

\subsection{Úlohy pro výpočet porušení v okolí tunelu, segmentace EDZ}

Bez uvažování podstatné heterogenity nebo poruchy podél liniového podzemního díla, můžeme uvažovat nulovou deformaci ve směru osy díla (tunelu) a tudíž řešení 2D úlohy s geometrií odpovídající řezu kolmo na osu. Můžeme uvažovat kruhový nebo obecný profil (viz Obr. \ref{Obr1}). Okrajové a počáteční podmínky
\begin{itemize}
	\item počáteční napětí v dané oblasti,
	\item nulové (normálové) posuny na vnější hranici,
	\item nulové napětí na vnitřní hranici (obvodu tunelu).
\end{itemize}

V případě, že počáteční napětí v dané oblasti zjistíme výpočtem podle obr.2, můžeme počítat napětí a deformace řešením úloh podle obr.3.

Následně zjistíme zóny porušení jedním z výše uvedených postupů (i)-(iv). Tedy jako zóny s překročením pevnostního kritéria při postupu (i), zóny plastických deformací při postupu (ii) nebo zóny s porušením při postupu (iii) nebo (iv). Zóny porušení můžeme dále segmentovat na zóny různého
stupně porušení.

Příklad uvedeného postupu lze nalézt v (Perras, Diedrichs \cite{18}) nebo v
(Blaheta a dal. \cite{13}). V \cite{18} jde o kval ... \alert{???}

Vstupní parametry úloh jsou určeny s určitým stupněm nejistoty a proto
je potřeba vyjasnit citlivost řešení (výstupů) na jednotlivé parametry a
provést zpřesnění důležitých parametrů pomocí měřených dat. Měřená data
mohou být určení stupně a hloubky porušení pomocí seismické a odporové
tomografie, nebo pomocí video karotáže ve vrtech směřujících vně zkoumaného
podzemního díla. Dalším typem měření mohou být konvergence, tedy
změny vzájemných vzdáleností zvolených bodů.

Řešené úlohy mohou uvažovat časový faktor jako je reologie hornin a
speciálně reakce při postupu ražby tunelů. V tomto případě 3D nebo 2D s
...\footnote{Rešerše}. \alert{???}

\subsection{Kalibrace modelů a inverzní úlohy}

Každý typ modelu vyžaduje specifické parametry a našim cílem bude jak popsat model, tak následně uvažovat citlivost na změny jednotlivých parametrů i možnost, jak parametry určit. Standardním způsobem určení parametrů jsou laboratorní zkoušky \alert{se vzorky hornin. Ty jsou ale rozměrově odlišné a více intaktní narozdíl od horninového masívu.} Vítaná je tedy možnost získání parametrů pomocí inverzních úloh a in situ měření.



\section{Proudění}

Mechanické porušení hornin v okolí podzemních chodeb vede ke změně
hydraulické vodivosti, a to v rozsahu až několika řádů. Změnu vodivosti
můžeme lokálně měřit pomocí hydraulických testů, ale další informaci nám
mohou poskytnout matematické modely. Proudění je určeno řadou vlivů:
\begin{itemize}
	\item skutečnosti, zda uvažujeme plně saturované prostředí (situace před a po určitou dobu po vyražení podzemního díla a dlouhodobá situace po utěsnění a resaturaci) nebo zda uvažujeme jen částečně saturované prostředí,
	\item velikost počátečního pórového tlaku vody (kapaliny),
	\item hydraulická vodivost neporušené horniny a případně existujících trhlin,
	\item velikost zón určitého stupně porušení.
\end{itemize}
Počáteční pórový tlak lze odhadnout obdobně jako vertikální počáteční napětí
vahou kapaliny, tedy $p_w = h\rho_w g$, kde ... Skutečný tlak se ale může od uvedeného odhadu lišit.

\subsection{Modely proudění}
V případě proudění v saturovaném horninovém prostředí (neporušený horninový masiv, masiv s otevřeným tunelem po určitou dobu od vytvoření tunelu, stav uzavřeného úložiště) použijeme k simulaci proudění Darcyho model, tedy
\begin{equation}\label{eq2}
	div(v) = Q, \quad  v = - K\nabla p_w \quad \mbox{v} \ \Omega,
\end{equation}
případně
\begin{equation}\label{eq3}
	c_s \frac{\partial p}{\partial t} = div (K\nabla p_w) + Q, \  \mbox{v} \ \Omega \times (0, t_{max}),
\end{equation}
kde $K$ je tenzor hydraulické vodivosti, v isotropním případě $K = \kappa I$, $c_s$ je
storativita, $Q$ je zdrojový člen.

Pro ustálený stav (\ref{eq2}) musíme zadat okrajové podmínky na hranici $\partial \Omega$,
pro neustálený stav musíme navíc zadat počáteční podmínku $u(x, 0) = u_0(x)$
pro $x \in \Omega$.

V případě proudění v částečně saturovaném prostředí můžeme proudění
popsat Richardsovou rovnicí, její tvar je obdobný, ale rovnice
\begin{equation}\label{eq4}
	c_m (p) \frac{\partial p}{\partial t} = div (k_r(p) K\nabla p_w) + Q, \  \mbox{v} \ \Omega \times (0, t_{max}),
\end{equation}
obsahuje dva nelineární členy, které závisí na na tlaku $p$. Ten je v případě
prostředí s částečnou saturací záporný a souvisí se saturací prostřednictvím
empiricky dané retenční křivky. Člen $c_m(p)$ je odvozen z retenční křivky, člen $k_r(p)$ s hodnotami mezi 0 a 1 je relativní permeabilita, která je rovna
1 v případě plné saturace.

Uvedené modely popisují spojité prostředí s možnou heterogenitou, kdy
především koeficient vodivosti $k = k(x)$ je různý v různých částech oblasti $\Omega$ \alert{[SS: Asi spíš $\kappa$ místo $k$.]}. Popsanými modely lze tedy popisovat proudění v prostředí se zónami
porušení vzniklými ražbou nebo existujícími už v ražbou neovlivněném stavu.
V~případě prostředí s trhlinami lze použít modely kombinující proudění v
oblastech různé dimenze, viz ...\footnote{reference}.

\subsection{Inverzní úlohy s apriori zadanými zónami různé vodivosti}
Hydraulická vodivost se v zájmové oblasti může měnit, a to v závislosti na
deformaci (napětí) a v závislosti na porušení horniny. Uvažujme nejdříve
proudění ovlivněné porušením horniny a nikoliv změnou napětí. Potom nejmenší vodivost přiřadíme neporušenému horninovému masivu (neporušená
hornina plus homogenizované poruchy), další zónám různého stupně porušení,
tedy jednotlivým částem EDZ. Přitom můžeme EDZ seqmentovat na řadu
částí (zón), nejen na obvyklé EDZ, HDZ (highly damaged zone), CDZ (construction
damaged zone), viz např. (Perraz, Diederichs \cite{18}). Výsledkem je
dělení oblasti na disjunktní části, $\Omega = \Omega_1 \cup \cdots \cup \Omega_m$, kde $k = k_i$ je konstantní v $\Omega_i$.

Inverzní úlohy uvedeného typu byly studovány v (Haslinger a dal. \cite{17}),
(Blaheta a dal. \cite{15}). V uvedených referencích byly uvažovány měření průměrného
vtoku/výtoku přes části hranice i měření odpovídající tlakovým zkouškám.
Je otázkou, jaká měření můžeme realizovat a použít pro určení vodivostí 
$k_i$ v případě analýzy vodivosti EDZ. Možnosti jsou průměrné toky přes
části stěny tunelu, pórové tlaky na stěně tunelu i měřené ve vrtech, to vše
při proudění daném přirozeným pórovým tlakem nebo při realizaci tlakové
zkoušky s injektáží vody.

Inverzní úloha potom spočívá v hledání minima cenové funkce $F$, která je
daná normou rozdílu mezi měřenými veličinami (vektor $\mu \in R^m$) a odpovídajícími
výstupy $Sp$, které jsou odvozeny z řešení $p = \mathcal{M}(k)$ modelu proudění $\mathcal{M}$
(viz část 3.1) s hydraulickými vodivostmi reprezentovanými složkami vektoru
$k \in R^s$,
$$
	F(k) = \frac{1}{2}\|S \mathcal{M} (k) - \mu\|^2
$$

V (Blaheta a dal. \cite{15}) je ukázáno, že ve standardní situaci s nepřesným
měřením nebo relativně přesným měřením, ale nepřesnostmi v samotném
popisu úlohy, může být řešení inverzní úlohy zatíženo velkou chybou. V
takovém případě je nutná regularizace cenové funkce, jejíž volba se většinou
řídí ad hoc postupy a není jednoduchá. Proto je zajímavé, i když výpočetně
náročné, použití techniky bayesovské inverze, viz např. (Blaheta a dal. \cite{15,16}).

Paolo Fabbri, Mirta Ortombina and Leonardo Piccinini

Estimation of Hydraulic Conductivity Using the Slug Test Method in a
Shallow Aquifer in the Venetian Plain (NE, Italy) Permeability or hydraulic
conductivity (K) is an essential parameter to understanding the movement of
groundwater and pollution. The slug test is a fast and inexpensive technique
for the determination of this fundamental value.

Měření stoupání hladiny ve studni

\section{Hydromechanika}
V případě situace, která neuvažuje zvláště vybrané poruchy nebo trhliny lze
využít klasický Biotův model. Podstatný vliv ale také může mít závislost hydraulické propustnosti na deformaci masivu. V případě modelů s trhlinami,
je podstatná závislost hydraulické propustnosti na rozevření trhlin.

\cite{16,30,29,31,32}


\subsection{Inverzní úlohy hydromechaniky}

V článku ... \alert{???}
$$
	\kappa = \kappa_i \mbox{exp}(\beta e_d),
$$
kde $\kappa_i$ je permeabilita neporušeného masivu, $\beta$ je konstanta, kterou je
potřeba určit experimentem nebo řešením inverzní úlohy, $e_d$ je ekvivalentní
deviatorická deformace,
$$
	e_d = \frac{2}{\sqrt{6}}[(\varepsilon_1 - \varepsilon_2)^2 + (\varepsilon_2 - \varepsilon_3)^2 + (\varepsilon_3 - \varepsilon_2)^2]^{1/2},
$$
kde $\varepsilon_1$, $\varepsilon_2$ a $\varepsilon_3$ jsou hlavní deformace.

Proti původnímu stavu, pokud ten je nula

Permeabilita může také záviset na velikosti objemové deformace $e_v = $

kombinovaně
$$
	\kappa = [\kappa_i + \kappa_0 \mbox{exp}(\beta_v e_v)]\, \mbox{exp}(\beta_d e_d).
$$

vztah permeabilita-deformace

Nejmenší čtverce

Bayes

\section{Stochastické přístupy, Bayesovská inverze}

[jen kopie existujícío textíku:] Při využití bayesovské techniky lze řešením
inverzní úlohy získat nikoliv určité řešení, ale statistickou informaci o řešení,
tedy o parametrech, které máme identifikovat. Taková formulace je přirozená
a robustní, předpokládá se nepřesné měření s danou statistickou charakteristikou
chyby měření a směřuje se k získání statistického popisu řešení inverzní
úlohy. Cílem je tedy odhadnout rozdělení pravděpodobnosti vektoru,
jehož složky jsou našem případě konstantní hydraulické vodivosti v daných
podoblastech. Bayesovská inverze se zaměřuje na charakterizaci vodivosti
$k \in U_{ad}$ reprezentované vektorem $k \in R^m$ nebo vektorem transformovaných
hodnot $\lambda = \ln(k) \in R^m$, pro které $\mu = \mathcal{G}(\lambda)+\eta$, kde $\mu$ je vektor měření, $\mathcal{G}$ je operátor pozorování, $\eta$ je šum měření. Bayesovská inverze používá apriorní statistické informace o vektoru $\lambda$ dané ve formě funkce hustoty (pdf) $\pi_0$ a statistické informace o šumu dané jeho společnou pdf $\pi_{\eta}$. Cílem je pak najít aposteriorní pdf $\pi(\lambda|\mu)$ vektoru $\lambda$ (a odpovídající vodivosti $k$) s nejistými
(náhodnými) materiálovými parametry. S ohledem na Bayesovu teorém
 $\pi(t|\mu) \propto \pi_{\eta}(\mu-G(t)) \pi_0 (t)$, kde $\propto$ označuje proporcionalitu. Numerická
realizace Bayesovské inverze používá vzorkování ze zadního rozdělení algoritmem
Metropolis-Hastings. Pro snížení extrémně vysokých výpočetních
nákladů používáme aproximaci $\tilde{G}$ k $G$ a zpožděný akceptační algoritmus
Metropolis-Hastings \cite{5}. Aproximace $\tilde{G}$ je zajištěna stochastickou Galerkinovou
metodou, viz \cite{5,6,7}.

A deeper insight into the inverse problem can be achieved by using the
Bayesian inverse approach \cite{4}. It assumes given statistical properties of
the measurement error and does not attempt to get deterministic material
characteristics, but attempts to obtain their statistical description, i.e.
to estimate the joint probability distribution of a vector representing the
piecewise constant hydraulic conductivity in our case. The Bayesian inverse
aims at characterization of the conductivity $k \in U_{ad}$ represented by a vector
$k\in R^m$ or vector of transformed values $t=\ln(k)\in R^m$ for which $\mu=G(t)+\eta$,
where $\mu=(m_{ij})$ is the vector of measurements, $G$ is the observation operator,
$G(t)=(q_{ij} (k))$ provides the vector of computed fluxes, $\eta$ is the measurement
noise. The Bayesian inversion uses a prior statistical information about the
vector $t$ given in the form of the joint probability density function (pdf) $\pi_0$
and statistical information about the noise given by its joint pdf $\pi_{\eta}$. The
aim is then to find the posterior pdf $\pi(t|\mu)$ of the vector $t$ (and correspondingly
$k$) with uncertain (random) material parameters. With respect to the
Bayes theorem $\pi(t|\mu)\propto\pi_{\eta}(\mu-G(t)) \pi_0(t)$, where $\propto$ denotes the proportionality.
Numerical realization of the Bayesian inversion use sampling from the
posterior distribution by the Metropolis-Hastings algorithm. To reduce the
extremely high computational cost we use approximation $\tilde{G}$ to $G$ and the
delayed acceptance Metropolis-Hastings algorithm \cite{5}. The approximation
$\tilde{G}$ is provided by the stochastic Galerkin method, see \cite{5,6,7}.

\section{Shrnutí, poznámky, plány výzkumu a využití získaných poznatků}

co fractures

komplexní úloha M-H-T, zde pukliny, i bentonitová výplň

\begin{thebibliography}{32}
	\bibitem{1}
	R. Pusch, Geological Storage of Highly Radioactive Waste. Current
	Concepts and Plans for Radioactive Waste Disposal. Springer 2008
	
	\bibitem{2}
	Krietsch, H., Gischig, V., Evans, K. et al. Stress Measurements for
	an In Situ Stimulation Experiment in Crystalline Rock: Integration
	of Induced Seismicity, Stress Relief and Hydraulic Methods. Rock
	Mech Rock Eng 52, 517-542 (2019). https://doi.org/10.1007/s00603-018-1597-8 
	
	\bibitem{3}
	Lubomír Staš, Josef Malík, Jan Franěk et al., Study of the stress conditions
	and internal anisotropy in an environment of granitic rocks. Institute
	of Geonics v.v.i., CAS Ostrava, Annual report Number 120/2017,
	December 2016
	
	\bibitem{4}
	B.H.G. Brady and E.T. Brown, Rock Mechanics for Underground Mining.
	3rd Edition, Springer, Dordrecht 2006
	
	\bibitem{5} C. Carranza-Torres, C. Fairhurst, The elasto-plastic response of underground
	excavations in rock masses that satisfy the Hoek-Brown failure
	criterion. International Journal of Rock Mechanics and Mining Sciences,
	Volume 36, Issue 6, 1999, Pages 777-809.
	
	\bibitem{6}	J. Rutqvist, A. Rejeb, M. Tijani, C.-F. Tsang, Analyses of coupled
	hydrological-mechanical effects during drilling of the FEBEX tunnel
	at GRIMSEL. In: Coupled Thermo-Hydro-Mechanical-Chemical Processes
	in Geo-Systems Fundamentals, Modelling, Experiments and Applications
	Edited by Ove Stephanson, Elsevier Geo-Engineering Book
	Series, Volume 2, (2004), pp. 131-136
	
	\bibitem{7} C.S. Desai, H.J. Siriwardane, Constitutive laws for engineering materials:
	With emphasis on geologic materials. Prentice-Hall, Englewood
	Cliffs, New Jersey 1984
	
	\bibitem{8} B. Singh, R. K. Goel, Engineering Rock Mass Classification.
	Butterworth-Heinemann, Elsevier Oxford 2011 pp. 171, 326
	
	\bibitem{9} Eduardo de Souza Neto, Djordje Peric, David Owens, Computational
	methods for plasticity: theory and applications. John Wiley \& Sons,
	Chichester 2008
	
	\bibitem{10} A. Zang, O. Stephansson, Stress Field of the Earth's Crust. Springer
	Dordrecht 2010
	
	\bibitem{11} Bukov
	
	\bibitem{12} a
	
	\bibitem{13} R. Blaheta et al., Analysis of \"{A}sp\"{o} Pillar Stability Experiment: Continuous thermo-mechanical model development and calibration. Journal of
	Rock Mechanics and Geotechnical Engineering 5 (2013) 124-135
	
	\bibitem{14} R. Blaheta, Numerical Methods in Elasto-Plasticity, Peres Publishers,
	1999
	
	\bibitem{15} Radim Blaheta, Michal Béreš, Simona Domesová, Pengzhi Pan, A comparison
	of deterministic and Bayesian inverse with application in micromechanics.
	Applications of Mathematics vol. 63(2018), issue 6, pp.	665-686.
	
	\bibitem{16} Radim Blaheta, Michal Béreš, Simona Domesová, D. Horák, Bayesian
	inversion for steady flow in fractured porous media with contact on
	fractures and hydro-mechanical coupling. Computational Geosciences
	(2020) 24:1911-1932
	
	\bibitem{17} Jaroslav Haslinger, Radim Blaheta, Raino A. E. M\"{a}akinen, Parameter
	identification for heterogeneous materials by optimal control approach
	with flux cost functionals. Mathematics and Computers in Simulation,
	on-line June 2020
	
	\bibitem{18} Matthew A. Perras, Mark S. Diederichs, Predicting excavation damage
	zone depths in brittle rocks. Journal of Rock Mechanics and Geotechnical
	Engineering 8 (2016), 60-74
	
	\bibitem{19} Jonny Rutqvist, Lennart Borgesson, Masakazu Chijimatsu, Jan Hernelind,
	Lanru Jing, Akira Kobayashi, Son Nguyen, Modeling of damage,
	permeability changes and pressure responses during excavation of
	the TSX tunnel in granitic rock at URL, Canada. Environ Geol (2009)
	57:1263-1274.
	
	\bibitem{20} Christoph Butscher, Steady-state groundwater inflow into a circular
	tunnel. Tunnelling and Underground Space Technology 32 (2012) 158-167
	
	\bibitem{21} Hadi Farhadian, Arash Nikvar-Hassani, Water flow into tunnels in discontinuous
	rock: a short critical review of the analytical solution of the art. Bulletin of Engineering Geology and the Environment (2019) 78:3833-3849
	
	\bibitem{22} John H. Black, Nicholas D. Woodman, John A. Barker, Groundwater
	flow into underground openings in fractured crystalline rocks: an
	interpretation based on long channels. Hydrogeol J (2017) 25:445-463
	
	\bibitem{23} I. Skarydova, M. Hokr, Modelling of tunnel in
ow with coupled 3D
	groundwater and 2D surface flow concept. Hydropredict 2010 Proceedings,
	Prague 2010
	
	\bibitem{24} Milan Hokr, Ilona Škarydová, Dalibor Frydrych, Modelling of tunnel in-
	flow with combination of discrete fractures and continuum. Computing
	and Visualization in Science, volume 15, pages 21-28(2012)
	
	\bibitem{25} In-Mo Lee, Seok-Woo Nam, Effect of tunnel advance rate on seepage
	forces acting on the underwater tunnel face. Tunnelling and Underground
	Space Technology 19 (2004) 273-281
	
	\bibitem{20} Christoph Butscher, Steady-state groundwater inflow into a circular
	tunnel. Tunnelling and Underground Space Technology 32 (2012) 158-167
	
	\bibitem{21} Hadi Farhadian, Arash Nikvar-Hassani, Water flow into tunnels in discontinuous
	rock: a short critical review of the analytical solution of the art. Bulletin of Engineering Geology and the Environment (2019) 78:3833-3849
	
	\bibitem{22} John H. Black, Nicholas D. Woodman, John A. Barker, Groundwater
	flow into underground openings in fractured crystalline rocks: an
	interpretation based on long channels. Hydrogeol J (2017) 25:445-463
	
	\bibitem{23} I. Skarydova, M. Hokr, Modelling of tunnel in
ow with coupled 3D
	groundwater and 2D surface flow concept. Hydropredict 2010 Proceedings,
	Prague 2010
	
	\bibitem{24} Milan Hokr, Ilona Škarydová, Dalibor Frydrych, Modelling of tunnel in-
	flow with combination of discrete fractures and continuum. Computing
	and Visualization in Science, volume 15, pages 21-28(2012)
	
	\bibitem{25} In-Mo Lee, Seok-Woo Nam, Effect of tunnel advance rate on seepage
	forces acting on the underwater tunnel face. Tunnelling and Underground
	Space Technology 19 (2004) 273-281
	
	\bibitem{26} Chungsik Yoo, Interaction between Tunneling and Groundwater. 				Numerical Investigation Using Three Dimensional Stress-Pore Pressure Coupled Analysis. Journal of Geotechnical and Geoenvironmental Engineering, Vol. 131, No. 2, February 1, 2005, pp. 240-250
	
	\bibitem{27} Yifeng Chen, Jiamin Hong, Shaolong Tang, Chuangbing Zhou, Characterization
	of transient groundwater flow through a high arch dam foundation during reservoir impounding. Journal of Rock Mechanics and Geotechnical Engineering 8 (2016) 462-471
	
	\bibitem{28} M. Hokr, H. Shao, W. P. Gardner, A. Balvín, H. Kunz, Y. Wang \& M. Vencl, Real-case benchmark for flow and tracer transport in the fractured rock. Environmental Earth Sciences volume 75, Article number: 1273 (2016)
	
	\bibitem{29} R. Blaheta, M. Béreš, S. Domesová, J. Haslinger, D. Horák, Inverse problems for identification of the hydraulic conductivity of rocks, fractures and excavation damage zones (EDZ). Sborník konference Geotechnika 2020, Beskydy Soláň, 10.-11.9. 2020 [TACR ENDORSE]
	
	\bibitem{30} R. Blaheta, M. Béreš, S. Domesová, Inverse problems for flow and hydromechanics
	in fractured geological media. CouFrac 2020 Conference Proceedings, [EURAD]
	
	\bibitem{31} Simona Domesová, Michal Béreš and Radim Blaheta, Efficient implementation
	of the Bayesian inversion by MCMC with acceleration of posterior sampling using surrogate models. Vyjde v IACMAG Proceedings, Springer 2020 (Proceeding) [TACR ENDORSE]
	
	\bibitem{32} Michal Béreš, Radim Blaheta, Simona Domesová, David Horák, Numerical methods for simulation of coupled hydro-mechanical processes in fractured porous media. Vyjde v IACMAG Proceedings, Springer 2020 (Proceeding) [EURAD]
\end{thebibliography}


\end{document}







	
	
		
		
		
		
		
		
